% rubber: module pdftex
% rubber: module index
% rubber: module xr
\documentclass[a4paper,american,12pt]{scrreprt}

\usepackage[margin=2.5cm]{geometry}

\usepackage{graphicx}
\usepackage[headsepline]{scrpage2}
\usepackage[utf8]{inputenc}
\usepackage[T1]{fontenc}
\usepackage{color}
\usepackage{caption}

%\usepackage{times}
%\usepackage{lmodern}
\usepackage{mathptmx}
% required for mathbb
\usepackage{amsmath}
\usepackage{txfonts}
%\usepackage[scaled=.9]{helvet}
%\usepackage{courier}

\usepackage{babel}
\usepackage{microtype}

\usepackage{listings}

\usepackage{hyperref}

\usepackage{todonotes}
\usepackage{xspace}

\automark[section]{chapter}

\parskip.5em
\parindent0em

\newcommand{\cinco}{\textsc{Cinco}\xspace}
\newcommand{\code}[1]{\texttt{#1}}
\newcommand{\newnotion}[1]{\emph{#1}}
\newcommand{\gui}[1]{\flqq{}\textsf{#1}\frqq{}}


\begin{document}


\chapter{Basics}

\section{About This Document}

This is the technical manual for the ``\cinco SCCE Meta Tooling Suite'' (in
short just \cinco). It is intended to provide a basic usage introduction as
well as reference for all features and keywords. It only implicitely covers the
conceptual ideas behind \cinco. For a more in-depth and scientific view on those,
please refer to the following publications:
%
\begin{itemize}
\item S. Naujokat et al.: Full generation of domain-specific graphical modeling tools: a
meta$^2$modeling approach \cite{NaLSKM2014}
\item S. Naujokat et al.: Domain-Specific Code Generator Modeling: A Case Study for
Multi-Faceted Concurrent Systems \cite{NaTISL2014}
\item D. Kopetzki: Model-based generation of graphical editors on
the basis of abstract meta-model specifications \cite{Kopetz2014}
\end{itemize}

This document will constantly be revised and extendend. The latest version will
always be available for download at \url{http://cinco.scce.info}. 

\begin{description}
\item[date of build] \today
\item[covered version] \cinco v. 0.4
\end{description}

\section{Introduction}

The ``\cinco{} SCCE Meta Tooling Suite'' is an integrated devlopment environment
(IDE) for the quick and easy development of domain-specific graphical modeling
tools. \cinco{} makes use of several frameworks from the Eclipse Modeling
Project, but its main goal is to hide those frameworks' intricacy from the
tool developer. \cinco{} provides two simple textual specification languages (MGL and
MSL, see below) for the high-level characterization of the developed tool's
structural and visual features. Above that, \cinco{} is also a full Java IDE, so
that the portions of the developed tool that are not described in a declarative
way can seamlessly be developed in the same environment.

\section{Download \& Installation}

As \cinco{} is a full Eclipse-based IDE including the Eclipse Modeling Tools
(EMF, Xtext, Graphiti, ...) and the Java Development Tooling (JDT) in addition
to our own plug-ins the download size is almost 300 MB. We are planning to
provide the \cinco{}-specific plug-ins via an Eclipse update site in the future,
but for now you need to download the whole package as one.

Please note: \cinco{} itself is developed under the ``Eclipse Public License v.
1.0'', but the executable build contains several libraries from the jABC
project, for which the following license applies:

\todo[inline]{include jABC license}

\cinco{} requires OpenJDK 7 to be installed on your system as main Java
installation (i.e. JAVA\_HOME pointing there). Apart from that, you just have to
choose the correct zip file for download (Linux/Window/Mac in 32 or 64 bit
variant), unzip it and execute the included binary: \code{cinco} for Mac and
Linux, \code{cinco.exe} for Windows.

\section{First Startup}

When starting \cinco{} it will first query you for the location of the workspace
directory. Choose some place on your harddisk that is not within the \cinco
installation folder. This workspace will contain all your \cinco projects
and the according metadata, as well as sub-projects, specification files, source
code etc. It is possible to use different workspaces (e.g. for different
projects). If you don't want to use multiple ones for now, you can check
\gui{Use this as default and do not ask again} before clicking \gui{OK}.

\bibliographystyle{alpha}
\bibliography{world}


\end{document}
